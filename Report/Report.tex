\documentclass{article}


% if you need to pass options to natbib, use, e.g.:
%     \PassOptionsToPackage{numbers, compress}{natbib}
% before loading neurips_2023


% ready for submission
\usepackage[preprint]{neurips_2023}


% to compile a preprint version, e.g., for submission to arXiv, add add the
% [preprint] option:
%     \usepackage[preprint]{neurips_2023}


% to compile a camera-ready version, add the [final] option, e.g.:
%     \usepackage[final]{neurips_2023}


% to avoid loading the natbib package, add option nonatbib:
%    \usepackage[nonatbib]{neurips_2023}


\usepackage[utf8]{inputenc} % allow utf-8 input
\usepackage[T1]{fontenc}    % use 8-bit T1 fonts
\usepackage{hyperref}       % hyperlinks
\usepackage{url}            % simple URL typesetting
\usepackage{booktabs}       % professional-quality tables
\usepackage{amsfonts}       % blackboard math symbols
\usepackage{nicefrac}       % compact symbols for 1/2, etc.
\usepackage{microtype}      % microtypography
\usepackage{xcolor}         % colors


\title{Unsupervised Domain Adaptation}


% The \author macro works with any number of authors. There are two commands
% used to separate the names and addresses of multiple authors: \And and \AND.
%
% Using \And between authors leaves it to LaTeX to determine where to break the
% lines. Using \AND forces a line break at that point. So, if LaTeX puts 3 of 4
% authors names on the first line, and the last on the second line, try using
% \AND instead of \And before the third author name.


\author{ Krishna Agarwal\\%\thanks{Use footnote for providing further information
    %about author (webpage, alternative address)---\emph{not} for acknowledging
    %funding agencies.} \\
  Indian Institute of Science, Bangalore\\
  \texttt{krishnaagarw@iisc.ac.in} \\
  % examples of more authors
   \And
  {Pratham Gupta} \\
  Indian Institute of Science, Bangalore\\
  \texttt{prathamgupta@iisc.ac.in} \\
   \And
   {Gavish Bansal} \\
   Indian Institute of Science, Bangalore\\
   \texttt{gavishbansal@iisc.ac.in} \\
   \And
   {Kintan Saha} \\
   Indian Institute of Science, Bangalore\\
   \texttt{kintansaha@iisc.ac.in} \\
  % \And
  % Coauthor \\
  % Affiliation \\
  % Address \\
  % \texttt{email} \\
}


\begin{document}


\maketitle


\begin{abstract}
  
\end{abstract}


\section{Introduction}
Unsupervised domain adaptation (UDA) is a type of domain adaptation in machine learning where a model is trained on a source domain with labelled data, and then adapted to a target domain with unlabelled data.
In UDA, the source domain and target domain have different distributions, but
the goal is to leverage the labelled data in the source domain to improve performance on the target
domain. \\
This report is product of our exploration of state of the art UDA algorithms and their applications in various fields like computer vision, natural language processing, etc. 
We have reproduced the results of these key papers [] in this research area of machine learning.

\section{Methodology}

\subsection{Algorithms}
We have implemented the following algorithms for our experiments:
\begin{itemize}
  \item \textbf{MMD (Maximum Mean Discrepancy)}[]: A metric that quantifies non-alignment between the source and target distributions. It is used as a loss function and in validating models. 
  \item \textbf{CORAL (Correlation Alignment), DeepCORAL}[][]: CORAL aligns the second-order statistics (covariances) of source and target features. Accordingly, DeepCORAL is an extension of CORAL that integrates correlation alignment into deep neural networks.
  \item \textbf{DANN (Domain-Adversarial Neural Network)}[]: Uses adversarial training to learn domain-invariant features. 
  \item \textbf{DSN (Domain Separation Network)}[]: Separates domain-specific and domain-invariant features for better adaptation. A state of the art method for domain invariant feature learning.
  \item \textbf{ATT (Adversarial Training with Triplet loss)}[]:An ensemble method that utilize two classifier trained on source domain to pseudo-label target domain to learn a classifier for it.
\end{itemize}

\subsection{Datasets}
We have used the following datasets for our experiments:
\begin{table}[h]
  \centering
  \caption{Benchmark datasets used in our experiments.}
  \label{tab:datasets}
  \begin{tabular}{ll}
      \toprule
      \textbf{Dataset Category} & \textbf{Datasets} \\
      \midrule
      Computer Vision (Numbers)     & MNIST, MNIST-M, SVHN \\
      Computer Vision (Categorical) & Office31, Amazon, DSLR, Webcam \\
      Sentiment Analysis (Classification)  & Amazon Review Sentiment \\
      \bottomrule
  \end{tabular}
\end{table}


\section{Experiments}
\section{Conclusion}

\section*{Appendix}



%%%%%%%%%%%%%%%%%%%%%%%%%%%%%%%%%%%%%%%%%%%%%%%%%%%%%%%%%%%%


\end{document}